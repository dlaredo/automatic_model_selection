%\runauthor{Laredo, \textit{et al.}}

\begin{frontmatter}

\title{Automatic Model Selection for Neural Networks}

\author{David Laredo$^{1}$, Yulin Qin$^{1}$, Oliver Sch\"utze$^{2}$ and Jian-Qiao Sun$^{1}$}
\address{
$^{1}$Department of Mechanical Engineering, School of Engineering\\
University of California, Merced, CA 95343, USA\\
$^{2}$Department of Computer Science, CINVESTAV, Mexico City, Mexico\\
Corresponding author: Jian-Qiao Sun. Email: jqsun@ucmerced.edu}

\begin{abstract}
Neural networks and deep learning are changing the way that artificial intelligence is being done. Efficiently choosing a suitable network architecture and fine tune its hyper-parameters for a specific dataset is a time-consuming task given the staggering number of possible alternatives. In this paper, we address the problem of model selection by means of a fully automated framework for efficiently selecting a neural network model for a given task: classification or regression. The algorithm, named Automatic Model Selection, is a modified is a modified micro-genetic algorithm that automatically and efficiently finds the most suitable neural network model for a given dataset. The main contributions of this method are: a simple list based encoding for neural networks as genotypes in an evolutionary algorithm, new crossover and mutation operators, the introduction of a fitness function that considers both, the accuracy of the model and its complexity and a method to measure the similarity between two neural networks. AMS is evaluated on two different datasets. By comparing some models obtained with AMS to state-of-the-art models for each dataset we show that AMS can automatically find efficient neural network models. Furthermore, AMS is computationally efficient and can make use of distributed computing paradigms to further boost its performance. 
\end{abstract}


\begin{keyword}
artificial neural networks\sep
model selection\sep
hyperparameter tuning\sep
distributed computing\sep
evolutionary algorithms
\end{keyword}

\end{frontmatter}

